\documentclass[a4paper,12pt]{report}
\usepackage[slovene]{babel}
\usepackage{listings}
\usepackage{graphicx}
\usepackage[section]{placeins}
\newcommand{\subtitle}[1]{%
  \posttitle{%
    \par\end{center}
    \begin{center}\large#1\end{center}
    \vskip0.5em}%
}

\lstset{
numberstyle=\small, 
numbersep=8pt, 
frame = single, 
language=SQL, 
framexleftmargin=15pt}

\begin{document}

\title{Skripta za pripravo na ustni izpit:\\Planiranje in upravljanje informatike}
\author{povzeto po predavanjih prof. Roka Rupnika}

\maketitle

\chapter{Uvodni sklop}

\paragraph{Informacijska družba} je družba v kateri je uspeh posameznika in organizacije odvisen od hitrosti procesiranja informacij in sposobnosti pridobiti pravo informacijo v pravem trenutku.

\paragraph{Informacijski sistem} opredelimo kot množico medsebojno odvisnih komponent,
   med katere spadajo strojna in programska oprema ter ljudje, ki zbirajo, procesirajo,
   hranijo in porazdeljujejo podatke in s tem podpirajo delavne procese v organizaciji.
      

\paragraph{Delitev IS} Informacijske sisteme delimo na \textbf{formalne} in \textbf{neformalne}. 
Nefromalne IS ne moremo natančno definirati, saj ne vemo kako delujejo \textit{(npr. instinkt)}. Takšnih IS ne moremo sprogramirati.

\section{Vrste informacijskih sistemov}

   \paragraph{Transakcijski IS} pokriva vsako dnevne dogodke/transakcije (CDR - call data register).

   \paragraph{Upravljalski (poslovodski) IS} omogočajo planiranje \textit{(npr. ali se na študijskem programu izplača povečati vpis)}

   \paragraph{Direktorski IS}\mbox{}

   \paragraph{Odločitveni IS} omogočajo svetovanje za sprejemanje odločitev, bazirajo na določenem modelu \textit{(npr. odobritev kredita na banki)}

   \paragraph{Ekspertni IS} \textit{poskušajo} nadomestiti eksperta \textit{(npr. diagnostika pacienta na podlagi njegovih simptomov)}, osnovani so na \textbf{bazi znanja} (\textit{if/else} pravila)

   \paragraph{Sistemi za avtomatizacijo/ Sistemi za podporo delovni procesom}\mbox{}

   \subsection{Dellitev IS na ravni}
      \paragraph{Operativna raven} je najnižja raven, ki vključuje predvsem transakcijske sisteme, kjer hranimo veliko količino neagregiranih podatkov.
      Stopnja avtomatizacije je visoka, IS na tej ravni so visoko formalni.
      \paragraph{Taktična raven} vmesna raven, kjer so podatki zmerno agregirani, še vedno je prisotna visoka stopnja formalnosti in avtomatizacije \textit{(npr. beleženje porabljenih enot v mobilnem paketu)}
      \paragraph{Stratežka raven} je najvišja ravem na kateri so prisotni večinoma agregirani podatki. Takšni sistemi so povečini neformalni (ali zgolj delno formalni), avtomatizacije je malo ali ni prisotna.

   \section{Življenjski modeli razvoja informacijskih sistemov}

   \paragraph{Faze razvoja IS} Razvoj IS običajno delimo v naslednje faze:
      \begin{enumerate}
         \item analiza
         \item načrtovanje
         \item implementacija
         \item testiranje
         \item uvedba
         \item vzdrževanje
      \end{enumerate}
   \paragraph{Življenjski model razvoja} \textit{(SDLC - system development life cycle)} pove sosledje in način izvedbe faz v okviru razvoja IS.

   \pagebreak

   \subsection{Pristopi k razvoju IS}
   \subsubsection{Zaporedni ali slapovni pristop}
   Ko se ena faza konča, se druga začne.

   \begin{center}
      \begin{tabular}{|c|c|}
         \hline
         \textbf{Faza} & \textbf{Cilj}\\
         \hline
         analiza & zahteve\\
         načrtovanje & načrt\\
         izvedba & koda\\
         testiranje&\\
         uvedba&IS\\
         \hline
      \end{tabular}
   \end{center}

   \paragraph{Prednosti in slabosti takšnega pristopa}\mbox{}
   \begin{center}
      \begin{tabular}{|c||c|}
         \hline
         \textbf{Prednosti} & \textbf{Slabosti}\\
         \hline
         malo režijskega dela & prepozno testiranje\\
         & nenaravno\\
         \hline
      \end{tabular}
   \end{center}

   \subsubsection{Iterativni pristop}
   Faze razvoja so razdeljene v iteracije, v vsaki iteraciji se izvedejo vse faze. Vsaka iteracija ima nek končni izdelek, ki je del celotne končne rešitve.

   \begin{center}
      \begin{tabular}{|c||c|}
         \hline
         \textbf{Prednosti} & \textbf{Slabosti}\\
         \hline
         povratne informacije & težko načrtovanje iteracij\\
         \hline
      \end{tabular}
   \end{center}

   \paragraph{Planiranje iteracij na makro ravni - release planing} celoten razvoj razdelimo na makro iteracije, vsaka izmed iteracij ima kot končen izdelek nek del informacijske rešitve.
   Te iteracije imenujemo \textbf{release} in so omejene na 2-4 mesece. Pred zadnjim relesom moramo imeti izdelek, ki vsebuje vse minimalne zahteve projekta
   \textbf{MVP - minimal value product}.

   \paragraph{Planiranje iteracij na mikro ravni - iteration planing} Makro iteracije razdelimo na manjše iteracije, ki trajajo med enim in dvema tednoma.
   V tem času želimo razviti del rešitve, ki sestavlja končen izdelek, ki ga želimo razvit v tem releasu. Iteracije delimo še na \textbf{task-e}.

   \subsubsection{Prototipni pristop}
   Temelji na delnih prototipnih izdelkih. Analizo in razvoj prototipa opravimo v začetku v celoti in kot rezultat pridobimo delovni prototip.
   Ne \textbf{delovnem prototipu} potem izvajamo testiranje in uporabo, preko katerih se odločamo o nadaljnem razvoju in izboljšavah.

   \subsubsection{Inkrementalni pristop}
   Inkrementalni pristop nam omogoča, da projekt razdelimo na več samostojnih podproblemov. Vsak izmed podproblemov je samostojen del končne informacijske rešitve.
   Vsak del razvijamo posebej ter ga ob zaključku predamo stranki v uporabo ter nadaljujemo z razvojem naslednjega podproblema.

   \paragraph{MoSCoW} - \textit{must, should, could, won't}

   \begin{center}
      \begin{tabular}{|c||c|}
         \hline
         \textbf{Prednosti} & \textbf{Slabosti}\\
         \hline
         na začetku rešimo najbolj tvegane sklope & vseh IS ne moremo razdeliti\\& na samostojne enote\\
         \hline
      \end{tabular}
   \end{center}

   
   
   
\end{document}